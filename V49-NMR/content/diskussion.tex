\section{Diskussion}
\label{sec:Diskussion}
\subsection{Relativer Fehler}
Alle relativen Fehler wurden nach der Formel
\begin{equation*}
  \tilde{x} = \frac{ \lvert x_{lit} - x_{mess} \rvert}{\lvert x_{lit} \rvert}
  \cdot 100 \%
\end{equation*}
berechnet, dabei bezeichnet $x_{lit}$ den Literaturwert der Messgröße $x_{mess}$.
\subsection{Relaxationszeiten \texorpdfstring{$T_1 \; \& \; T_2$}{math}}
Es fällt auf, dass die Werte von T1 und T2 nicht gleich groß sind. Es gilt $T_1 > T_2$ für beide
ist der Literaturwert von \SI{2,5}{second} gegeben \cite{hyper}. Die Quelle impliziert, dass dies üblich ist. Als
relative Abweichung ergibt sich für $T_1$ \SI{8,5}{\percent} und für $T_2$ \SI{34,2}{\percent}. Da es üblicherweise auftritt,
dass $T_2$ kleiner ist, sollte um eine Aussage über die Quälität der Messung zu machen, der relative Fehler von $T_1$
betrachtet werden. Dieser ist hinreichend genau unter den gegeben Umständen. Fehler sind damit zu erklären, dass die
Füllmenge der Probe nicht der, der vorgesehenen Halterung entspricht. Zudem können durch bewegen der Probe Störungen
entstehen.
\subsection{Vergleich der Molekülradien}
Der Vergleich der Molekülradien dient ebenfalls dazu die bestimmten Werte für die Diffusionskonstante und
die Viskosität zu vergleichen. Dies wird erreicht in dem der Stoksche Molekülradius mit den anderen beiden verglichen wird.
Der Vergleich mit $r_{hcp}$ liefert eine relative Abweichung von \SI{35}{\percent} und mit $r_{krit}$ eine Abweichung von
\SI{41,9}{\percent}. Da die Diffusionskonstante mit ähnlichen Methoden gemessen wurde wie die Relaxationszeiten sind hier
ähnliche Abweichungen zu erwarten, weshalb hier davon ausgegangen werden kann, dass die Viskositätsmessung nicht
hinreichend genau ist. Hier könnten apparative Ungenauigkeiten sowie menschliche Fehler entstanden sein, da die Messung
nur mit einer Stoppuhr durchgeführt werden. Durch mehrfache durchführen der Messung könnten diese anhand statistischer Methoden 
herausgerechnet werden und somit für ein genaueres Ergebnis sorgen.
