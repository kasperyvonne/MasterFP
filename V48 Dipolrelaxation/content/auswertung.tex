\section{Auswertung} \label{sec:Auswertung}
Im Rahmen des Versuches wurde der Relaxationsstrom gemessen, dieser ist bis auf einen konstanten 
Faktor proportional zur Stromdichte.
Deshalb können die zuvor dargestellten Zusammenhänge untersucht werden,
ohne das darauf Rücksicht genommen werden muss.
Die Daten zu den jeweiligen Diagrammen finden sich im Anhang \ref{sec:Anhang}.
\paragraph{Bestimmung und Abzug des Untergrundes}
In den Abbildungen \ref{fig:U2plot},\ref{fig:U15plot} sind die Relaxationsströme gegen die
Probentemperaturen aufgetragen, diese wurden mit verschiedenen Heizraten aufgenommen.
Es wurde versucht eine Messreihe mit einer Heizrate von \SI{2}{\celsius\per\minute} und eine
mit Heizrate \SI{1.5}{\celsius\per\minute} aufzunehmen.
Die berechneten, also die tatsächlichen Heizraten betragen
bei der Messreihe mit Soll-Heizrate \SI{2}{\celsius\per\minute},
$\SI{2.089}{\celsius\per\minute} = \SI{0.035}{\kelvin\per\second}$ und bei der
Messreihe mit \SI{1.5}{\celsius\per\minute} Soll-Heizrate,
$\SI{1.37}{\celsius\per\minute} = \SI{0.023}{\kelvin\per\second}$.
Zur Beginn der Auswertung wird der apparative Vorlauf des
Amperemeter von den Daten abgezogen. Danach werden die Minima der Relaxationskurven ausfindig
gemacht (rote Punkte),
da diese den Verlauf des Untergrundes abzeichnen. Der Untergrund nimmt exponentiell mit
der Temperatur zu. Deshalb wird dieser mit der Gleichung
\begin{equation}
U(T) = 	A\cdot \exp (C\cdot T ) + B
\end{equation}
angenährt und mit nicht-linearer Ausgleichsrechnung die Parameter $A,B,C$ bestimmt.
Die Parameter sind für die jeweiligen Messreihen in der Tabelle \ref{tab:Uparam} dargestellt.
Die Untergrundkurven sind auch in den Abbildungen \ref{fig:U2plot},\ref{fig:U15plot} zu finden.
Zudem wurden in den Abbildungen der Bereich der den ersten Relaxationsprozess zeigt eingeschränkt
(rote Linien), nur dieser Bereich wird in der späteren Auswertung betrachtet.
Außerdem ist das Maximum der
Relaxationskurve eingezeichnet, dieser Punkt ist ebenfalls für die spätere Auswertung interessant.
Nachdem der Untergrund bestimmt ist wird dieser von den Daten abgezogen und nur noch der
eingeschränkte Bereich betrachtet.

\begin{table}
 \centering
 \caption{Untergrund Parameter}
 \begin{tabular}{c| S@{${}\pm{}$} S | S@{${}\pm{}$} S}
   \toprule
    Heizrate &
    \multicolumn{2}{c}{$2 / \si{\celsius\per\minute}$} &
    \multicolumn{2}{c}{$1.5/ \si{\celsius\per\minute}$} \\
   \midrule
	A & 8&5&0&7e+02\\
	B&-4&5&0&7e+02\\
	C &0.02&0.01&0.001&0.02\\
   \bottomrule
 \end{tabular}
 \label{tab:Uparam}
\end{table}

\begin{figure}
  \centering
  \includegraphics[height = 10cm]{plots/M1UGplot.pdf}
  \caption{Darstellung der Daten die mit Heizrate \SI{2}{\celsius\per\minute} aufgenommen wurden.}
  \label{fig:U2plot}
\end{figure}
\begin{figure}
  \centering
  \includegraphics[height = 10cm]{plots/M2UGplot.pdf}
  \caption{Darstellung der Daten die mit Heizrate \SI{1.5}{\celsius\per\minute}aufgenommen wurden.}
  \label{fig:U15plot}
\end{figure}
\FloatBarrier
\paragraph{Bestimmung der Akivierungsenergie anhand der Nährung der Stromdichte}
Der folgende Teil der Auswertung bezieht sich auf die Nährung \eqref{eq:StromdichteNäherung2}
der Stromdichte. Durch logarithmieren dieser Gleichung ergibt sich folgender Ausdruck:
\begin{equation}
\ln(i(T)) = - \frac{W}{kT} + const  =  A \cdot \frac{1}{T} + B	\; .
\label{eq:lnfit}
\end{equation}
Dabei ist zu beachten, dass diese Nährung nur im Anfangsbereich der Kurve gültig ist.
In dieser Auswertung wurden die Bereiche des ersten Minimums bis zum Maximums der Relaxationskurve
verwendet, die in den Abbildungen \ref{fig:U2plot},\ref{fig:U15plot} zuvor eingezeichnet waren.
Die Parameter sind in der Tabelle \ref{tab:ln1param} dargestellt. Die Daten mit Ausgleichsgerade sind
in der Abbildung \ref{fig:Meth1} dargestellt. Aus den Werten lässt sich die Aktivierungsenergie $W$ berechnen
mit $A\cdot (-k)= W$. Die Werte dazu sind ebenfalls in der Tabelle \ref{tab:ln1param} dargestellt.
\begin{table}
 \centering
 \caption{Parameter der Ausgleichsgeraden}
 \begin{tabular}{c| S@{${}\pm{}$} S | S@{${}\pm{}$} S}
   \toprule
    Heizrate &
    \multicolumn{2}{c}{$2 / \si{\celsius\per\minute}$} &
    \multicolumn{2}{c}{$1.5/ \si{\celsius\per\minute}$} \\
   \midrule
	A &-1.1e+04&0.1e+04&-1.14e+04&0.08e+04\\
	B&46&4&46&3\\
	W / \si{\joule} &1.5e-19&0.1e-19&1.6e-19&0.1e-19\\
   \bottomrule
 \end{tabular}
 \label{tab:ln1param}
\end{table}

\begin{figure}
  \centering
  \includegraphics[height = 10cm]{plots/1.MethFitW.pdf}
  \caption{Darstellung der Daten im Gültigkeitsbereich der Nährung.}
  \label{fig:Meth1}
\end{figure}
\FloatBarrier
\paragraph{Bestimmung der Aktivierungsenergie und der charakteristischen Relaxationszeit}
Zuerst wird die Gleichung \eqref{eq:tau2} betrachtet. Darin wird die Definition der Relaxationszeit
\eqref{eq:Relaxationszeit} eingesetzt, nun wird die Gleichung logarithmiert und es ergibt sich folgender
Zusammenhang:
\begin{equation}
\ln \left(\frac{\int_T^{T^*} i(T') \symup{d}T'}{b \cdot i(T)} \right) =
\frac{W}{kT} + \ln \left(\frac{1}{\tau_0} \right)
= \frac{W}{kT} + const \rightarrow A \cdot \frac{1}{T} + B = \eqref{eq:lnfit}
\quad \text{mit} \quad i(T^*) \approx 0 \; .
\label{eq:lnint}
\end{equation}
Hier kann also wieder mit der linearen Ausgleichsrechnung \eqref{eq:lnfit} gearbeitet werden.
Dazu wird
\begin{equation*}
	\ln \left(\frac{\int_T^{T^*} i(T') \symup{d}T'}{b \cdot i(T)} \right)
\end{equation*}
gegen \sfrac{1}{T} aufgetragen, wie in Abbildung \ref{fig:Meth2} dargestellt. Die dadurch erhaltenen
Werte für die Parameter sind in Tabelle \ref{tab:ln2param} dargestellt. Aus den Parametern lassen sich dann
die Größen
\begin{equation*}
 W = A\cdot k \qquad \text{und} \qquad \tau_0 = \exp(-B)
\end{equation*}
berechnen, diese sind ebenfalls in der Tabelle \ref{tab:ln2param} dargestellt.

\begin{figure}
  \centering
  \includegraphics[height = 10cm]{plots/2.MethFitW.pdf}
  \caption{Darstellung der Daten nach Gleichung \ref{eq:lnint}.}
  \label{fig:Meth2}
\end{figure}

\begin{table}
 \centering
 \caption{Parameter der Ausgleichsgeraden}
 \begin{tabular}{c| S@{${}\pm{}$} S | S@{${}\pm{}$} S}
   \toprule
    Heizrate &
    \multicolumn{2}{c}{$2 / \si{\celsius\per\minute}$} &
    \multicolumn{2}{c}{$1.5/ \si{\celsius\per\minute}$} \\
   \midrule
	A &1.31e+04&0.05e+04&1.20e+04&0.04e+04\\
	B&-45&2&-40&1\\
	$W$ / \si{\joule}&1.81e-19&0.07e-19&1.66e-19&0.05e-19\\
	$\tau$ / \si{\second} & \multicolumn{2}{S}{3.9e+19} & \multicolumn{2}{S}{4.6e+17}\\
   \bottomrule
 \end{tabular}
 \label{tab:ln2param}
\end{table}
