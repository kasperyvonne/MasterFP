\section{Theorie}
\label{sec:Theorie}
Licht kann als elektromagnetische Welle interpretiert werden. Durch die Überlagerung
zweier Wellen treten unter bestimmten Bedingungen Interferenzeffekte auf, welche mit
einem Interferometer vermessen werden können. Damit lässt sich die in diesem Versuch
zu Untersuchende Größe des Brechungsindex untersuchen.
Eine Lichtwelle kann also wie in Gleichung \ref{eq:Welle} als Funktion von Raum und Zeit
mathematisch ausgedrückt werden. Dabei beschreibt der Vektor $\vec{E_0}$ die Polarisation des Lichtes an.

\begin{equation}
	\vec{E}(\vec{r},t) = \vec{E_0}\cdot \exp i(\omega t - \vec{k}\vec{r})
\label{eq:Welle}
\end{equation}

Der verwendete HeNe-Laser liefert kohärentes, also interferenzfähiges Licht.
Nach dem Superpositionsgesetz addieren sich zwei Wellen in jedem Raumzeitpunkt,
sodass eine überlagerte Welle entsteht. Wenn zwischen den Wellen ein Gangunterschied
$\delta$ existiert treten Interferenzeffekte. In einem Michelson-Interferometer wird der
Gangunterschied über die verschieden langen Laufwege der zwei Lichtstrahlen erzeugt, dies ist
beim Sagnac-Interferometer nicht der Fall. Hier wird der Gangunterschied erzeugt, indem
einer dier Lichtstrahlen die zuvermessende Probe durchläuft.  

\subsection{Funktionsweise des Sagnac-Interferometers}
