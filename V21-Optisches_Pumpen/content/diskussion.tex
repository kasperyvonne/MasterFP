\section{Diskussion}
\label{sec:Diskussion}
Die vertikale Erdmagnetfeldkomponente wurde mit Augenmaß eleminiert, in dem
die Breite des Nullpeaks minimiert wurde. Dabei wurde ein Magnetfeld der Stärke
$B_{\text{vertikal}}=35,40\,\mu$T eingestellt. Der Literaturwert der vertikalen
Komponente ist $B_{\text{vertikal}}=45,11\,\mu$T \cite{Mag}. Demnach entsteht
dabei eine Abweichung von 21\%.\\
Aus den $y-$Achsenabschnitten der Ausgleichsgeraden von \ref{fig:BFelder} lässt sich auf
die Horizontalkomponente des Ergmagnetfeldes schließen. Es zeigt sich, dass
die beiden Ausgleichsgeraden den selben $y-$Achsenabschnitt haben.
Die Werte weichen von den Literaturwerten ab, wie in Tabelle \ref{tab:Kernabweichung} dargestellt.
\begin{table}[H]
    \centering
    \caption{Darstellung der Kernspins der beiden Rubidiumisotope sowie den Literaturwerten aus \cite{Chem} und
             relativen Abweichungen.}
    \label{tab:Kernabweichung}
    \begin{tabular}{c|c|c|c}
        \toprule
        Isotop & Kernspin & Literaturwert  & Abweichung  \\
        \midrule
        $\symup{^{87}Rb}$ &(1,31 \pm 0,11)&$\frac{3}{2}$ & 13\%\\
        $\symup{^{85}Rb}$ & (2,35 \pm 0,10)&$\frac{5}{2}$& 6\%\\
        \bottomrule
    \end{tabular}
\end{table}
Das Verhältnis der Isotopen aus den Amplituden beträgt nach unserer Bestimmung
$r=(0,47 \pm 0,01)$, dies weicht um 28\% von dem Literaturwert von $r_{\text{Lit}}=0.386$ ab \cite{Peri}.
Dies könnte durch Fehler beim Ablesen erklärbar sein, da die Bilder des Oszilloskops keine
hohe Auflösung besitzen. \\
Der Literaturwert $\frac{b(\symup{^{87}Rb, Lit})}{b(\symup{^{85}Rb,Lit})} = 1.5$\cite{Anleitung} weicht um
 20\% von dem von uns bestimmten Wert von $\frac{b(\symup{^{87}Rb})}{b(\symup{^{85}Rb})}=(1,2\pm0,3)$ ab.\\
Die Abweichungen zu den Literaturwerten lassen sich mit hoher Wahrscheinlichkeit darauf zurückschließen, dass
 neben dem Versuchsaufbau der Versuch \textit{Faraday-Effekt} durchgeführt wurde. Bei diesem werden Magnetfelder
 von bis zu einem Tesla an und wieder abgeschaltet. Da die Messung des optischen Pumpen auf Einstellungen im $\mu$T Bereich beruhen
ist anzunehmen, dass die Änderungen des stärkeren Magnetfeldes in der Nähe der Apparatur die Messungen gestört hat.
