\section{Diskussion}
\label{sec:Diskussion}
Alle relativen Fehler wurden nach der Formel
\begin{equation*}
  \tilde{x} = \frac{ \lvert x_{lit} - x_{mess} \rvert}{\lvert x_{lit} \rvert}
  \cdot 100 \%
\end{equation*}
berechnet, dabei bezeichnet $x_{lit}$ den Literaturwert der Messgröße $x_{mess}$. 
\paragraph{Polarisationswinkel}
Der Polarisationswinkel für den Polarisationsfilter wurde zu \SI{48}{\degree} bestimmt, da der 
Polarisationsfilter aufgrund der vorhandenen Markierungen nur für Winkel 
die vielfache von \SI{5}{\degree} sind genau einstellbar ist, wurde 
dieser ausreichend genau bestimmt.
\paragraph{Brechungsindex des Plätchens} 
Der Brechungsindex des Plätchens wurde zu \\$\bar{n} = \SI{1.541(6)}{}$ bestimmt, aus 
\cite{Anleitung} ist bekannt, dass ein typisches Glas einen Brechungsindex von \SI{1.5}{} hat. 
Zum Vergleich wird von den Werten 1 subtrahiert. 
Damit ergibt sich eine relative Abweichung von ca. \SI{8.2}{\percent}. Abweichungen können durch eine 
mögliche Verschmutzung des Glases erklärt werden. Zudem sollte beachtet werden, dass alle Werte anhand einer 
Nährung berechnet wurden. Die Messung ist dennoch hinreichend genau.
\paragraph{Brechungsindex von Luft} 
Der Literaturwert für den Brechungsindex von Luft bei Normalbedingungen wurde aus 
\cite{brechlit} entnommen und beträgt $n_{lit} = \SI{1.000276638}{}$ für eine Wellenlänge von 
\SI{6236.10}{\angstrom}. Der experimentell bestimmte Wert liegt bei \SI{1.000302(28)}{}. Zum Vergleich 
wird wie zuvor von den Werten 1 subtrahiert.
Daraus folgt eine relative Abweichung von ca. \SI{9}{\percent}. Abweichungen können damit erklärt werden, dass 
der Literaturwert nicht für die gleiche Wellenlänge ermittelt wurde. Die Messwerte wurde hier mit einer 
Wellenlänge von \SI{632.990}{\nano\meter} aufgenommen. Im Rahmen in den die Messungen durchgeführt wurden, 
ist dies ausreichend genau. 
