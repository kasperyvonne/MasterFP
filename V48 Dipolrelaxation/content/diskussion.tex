\section{Diskussion}
\label{sec:Diskussion}
Alle relativen Abweichungen wurden nach der Formel
\begin{equation*}
  \tilde{x} = \frac{ \lvert x_{lit} - x_{mess} \rvert}{\lvert x_{lit} \rvert}
  \cdot 100 \%
\end{equation*}
berechnet. Der Vergleich der Methoden zur Bestimmung der Aktivierungsenergie wird unternommen, indem 
die relative Abweichung der Aktivierungsenergien für die jeweiligen Heizraten, bestimmt wird.
Die Aktivierungsenergie wurde für die Heizrate \SI{2}{\celsius\per\minute} mit der Nährungsmethode 
zu \SI{1.5(1)e-19}{\joule} und mit der Integralmethode zu \SI{1.81(7)e-19}{\joule} bestimmt. 
Daraus ergibt sich eine relative Abweichung von \SI{18}{\percent}. Für die Heizrate 
\SI{1.5}{\celsius\per\minute} ergab sich \SI{1.5(1)e-19}{\joule} und \SI{1.66(5)e-19}{\joule}. Damit ergibt 
sich eine relative Abweichung von \SI{6}{\percent}. Es ist zu sehen, dass die bestimmten Werte 
der Integralmethode größer ausfallen. Dabei ist noch einmal zu erwähnen, dass es sich bei der ersten 
Methode nur um eine Nährung handelt, es ist also zu erwarten, dass es zu Abweichungen kommt.\\
Die Integralmethode ermöglicht die charakteristische Relaxationszeit zu bestimmen, diese wurde 
für die Heizrate \SI{2}{\celsius\per\minute} zu \SI{3.8e+19}{\second} und für die Heizrate 
\SI{1.5}{\celsius\per\minute} zu \SI{4.5e+17}{\second} bestimmt. Die Werte liegen eine Größenordnung 
auseinander, weshalb sich eine Bestimmung der relativen Abweichungen erübrigt. \\
Abweichungen werden dadurch erklärt, dass die Heizrate nicht immer konstant gehalten werden kann. 
Besonders die ersten aufgenommenen Werte sind davon betroffen, da sich die Probe besonders schnell 
erwärmt, wenn der Kühlfinger aus dem Stickstoff genommen wird. Zudem wird ein sehr geringer Strom 
gemessen, der durch magnetische und elektrische Felder gestört werden kann. Ist der Kondensator der in 
der Apparatur verbaut ist nicht vollständig entladen, kommt es deshalb zu systematischen Fehlern. 

