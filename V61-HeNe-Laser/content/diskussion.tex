\section{Diskussion}
\label{sec:Diskussion}
\paragraph{Überprüfung der Stabilitätsbedingung}
In dem Versuchsteil sollten die theoretisch berechneten
Stabilitätsbedingung für unterschiedliche Spiegelkonfigurationen über die
Resonatorlängen überprüft werden.\\
Für beide Konfigurationen wurde der theoretische Wert nicht erreicht.
\begin{table}
  \centering
  \caption{Vergleich der gemessenen Resonatorlängen mit den Theoriewerten.}
  \label{tab:DisRe}
  \begin{tabular}{c|c | c | c}
      \toprule
      Konfiguration & $L_{\text{theo}}$ in cm & $L_{\text{max}}$ in cm  & relative Abweichung\\
      \midrule
      1 & 140 & 116,7 &  17 \% \\
      2 & 250 & 191,7 & 23 \%\\
      \bottomrule
  \end{tabular}
\end{table}
Das die berechneten maximal Werte nicht erreicht wurden, liegt möglicherweise an
einer ungenauen Justage des Lasers. Auch die Empfindlichkeit des
Versuchsaufbaus gegenüber kleinster Verrückungen senkrecht zur Ausbreitungsrichtung
kann das Aussetzten der Lasertätigkeit herbeigeführt haben.
\paragraph{Vermessung der TEM Moden}
Für die Grundmode TEM$_{00}$ lassen sich die Messwerte mit sehr geringen Fehlern
durch  die Theoriekurve beschreiben. Die Vermessung der Mode TEM$_{01}$ zeigt
jedoch eine starke Asymmetrie auf, welche
durch die angenommene Theoriekurve der Intensitäten nicht beschrieben wird.
Dies könnte an dem eingeschobenen Draht liegen, da dieser sehr verbogen war.
Oder bei der Messung hat sich nicht die reine Grundmode sondern eine
Überlagerung verschiedener Moden eingestellt. Eine
weitere allgemeine Fehlerquelle ist die Fluktuation der Laserintensität
durch Verrückungen am Versuchsaufbau, sowie das nicht vollständig
konstante Restlicht im Versuchsraum. Durch das Wechseln der Größenordnungen des
Ampermeters können weitere systematische Fehler aufgetreten sein.
\paragraph{Bestimmung der Polarisation}
Bei der Polarisationsmessung zeigten sich die starken Fluktuationen der Laserintensität
und die Problematiken des Ampermeters sehr deutlich. Die angezeigten Werte
änderten sich zu schnell um die Werte sicher ablesen zu können. Die Ausreißer
in den Messwerten lassen sich also durch ungenaues Ablesen erklären. Jedoch
konnte eine Abhängigkeit der Intensität nach dem Gesetz von Malus in guter
Näherung gezeigt werden, also die lineare Polarisation des Lichts gezeigt
werden.
\paragraph{Bestimmung der Wellenlänge}
Die mit dem Experiment bestimmte Wellenlänge ist $\lambda = (646 \pm 27)$ nm.
Die Theoretische Wellenlänge ist $\lambda = 632,8$ nm \cite{Anleitung}. Damit passt die bestimmte
Wellenlänge im Bereich der Fehlertoleranz zum Theoriewert.
\paragraph{Multimodenuntersuchung}
Für die drei vermessenen Resonatorlängen liegen die mittleren
Frequenzdifferenzen im MHz Bereich. Damit liegen die Abstäde der Moden weit
unterhalb der Aufweitung der Frequenzen durch den Dopplereffekt, welcher im
GHz bereich liegt. Die longitudinalen Moden treten als Schwebung im
Multimodenbetrieb des Lasers auf.
