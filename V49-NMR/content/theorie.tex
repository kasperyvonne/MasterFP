\section{Theorie}
\label{sec:Theorie}
\subsection{Ziel}
Mit gepulster Kernspinresonanz (NMR) kann der zeitliche Verlauf der Magnetisierung
einer Probe, unter Einwirkung eines Hochfrequenzfeldes untersucht werden. Die
dabei auftretenden Relaxationsprozesse werden unter Verwendung zweier
charakteristischer Relaxationszeiten beschrieben.
Aus diesen Größen wird die Diffusionskonstante von Wasser bestimmt.
\subsection{Magnetisierung einer Probe}
Durch ein Magnetfeld $\vec{B_0}=B_0 \vec{e_\text{z}}$, hier in Richtung der z-Achse,
spalten die Energieniveaus
in $2I+1$ äquidistante Unterniveaus auf. Dabei beschreibt $I$ die Spinquantenzahl.
Mit der Orientierungsquantenzahl $m$ mit $-I \leq m \leq I$ kann zwischen
den Zuständen unterschieden werden. Wenn sich die Probe im thermischen
Gleichgewicht mit der Umgebung befindet, sind die Niveaus gemäß der Boltzmannverteilung
unterschiedlich besetzt. Es ergibt sich eine mittlere Kernspinpolarisation, die
für Protonen mit $I=1/2$  und unter der Annahme $m \gamma B_0 \hbar \ll k_\text{B} T$ mit
\begin{equation}
  \langle I_{\text{Z}} \rangle= -\frac{\hbar^2}{4}\frac{\gamma B_0}{k_\text{B} T}
\end{equation}
genähert werden kann. Hier bezeichnet $\gamma$ das gyromagnetische Verhältnis des Kerns,
$k_\text{B}$ die Boltzmann-Konstante, $T$ die Temperatur
und $\hbar$ das Plancksche Wirkungsquantum.
Durch die Kernspinpolarisation wird über die magnetischen Momente $\vec{\mu_\text{I}}$ eine Magnetisierung
der Probe erzeugt. Der Betrag der Magnetisierung in Richtung des Magnetfelds ist
\begin{equation}
  M_0 = \frac{1}{4} \mu_0 \gamma^2 \frac{\hbar^2}{k_\text{B}} N \frac{B_0}{T},
\end{equation}
wobei $\mu_0$ die Permeabilität des Vakuums und
$N$ die Anzahl der Momente pro Volumeneinheit bezeichnet.\\
Um das zeitliche Verhalten der Probenmagnetisierung zu untersuchen, kann klssisch
Vorgegangen werden, da die Anzahl an Einzelmomenten $N$ mit etwa $10^{28}/$m$^3$
groß genug ist.
Auf die Probenmagnetisierung $\vec{M}$ wirkt im externen Magnetfeld $\vec{B_0}$ ein Drehmoment
$\vec{D}= \vec{M}\times \vec{B_0}$.  Dies führt zu einer Präzessionsbewegung um
die Richtung des Magnetfelds. Die Frequenz dieser Präzessionsbewegung wird
Larmor-Frequenz genannt und mit
\begin{equation}
  \omega_{\text{L}}= \gamma B_0 .
\end{equation}
Wird die Probenmagnetisierung durch ein Äußeres Feld aus ihrer Gleichgewichtslage entfernt, relaxiert sie nach
Ende der Störung wieder in die Ausgangslage zurück. Die Relaxion
wird über die Blochschen-Gleichungen
\begin{equation}
  \begin{split}
    \frac{\symup{d} M_\text{z}}{\symup{d} t} &= \frac{M_0 - M_\text{z}}{T_1} \\
    \frac{\symup{d} M_\text{x}}{\symup{d} t} &=
      \gamma B_0 M_\text{y} - \frac{M_\text{x}}{T_2} \\
    \frac{\symup{d} M_\text{y}}{\symup{d} t} &=
      - \gamma B_0 M_\text{x} - \frac{M_\text{y}}{T_2},
  \end{split}
  \label{eq:Bloch-Gleichungen}
\end{equation}
beschrieben.
Die auftretenden charakteristischen Zeitkonstanten $T_1$ und $T_2$ beschreiben dabei unterschiedliche
Effekte.  Die  sogenannte longitudinale oder
Spin-Gitter-Relaxationszeit $T_1$ bezeichnet die Veränderung des Magnetfelds parallel
zur Achse des Magnetfeldes. Sie beschreibt auch die Zeit in der bei Festkörpern Energie aus dem Kernspinsystem
in Gitterschwingungen umgewandelt wird. Die Bezeichnung wird auch bei flüssigen Proben beibehalten.
Die Zeitkonstante $T_2$ wird transversale oder Spin-Spin-Relaxationszeit bezeichnet. Diese Größe
beschreibt die Änderung der senkrechten Komponente der Probenmagnetisierung zur
Richtung des externen magnetischen Feldes. Durch Wechselwirkungen der Spins mit ihren
Nachbarn nimmt die Magnetisierung senkrecht zu $\vec{B}$ ab. \\
Um die Probenmagnetisierung aus dem Gleichgewicht zu entfernen wird
ein Hochfrequenzfeld $\vec{B_{\text{HF}}}$ senkrecht zu $\vec{B_0}$ angelegt. Dieses kann
in zwei zirkular polarisierte Felder mit den Frequenzen $\pm \omega$ zerlegt werden. Liegt die positive
Frequenz in der nähe der Larmorfrequenz, so kann der Einfluss des anderen Feldes vernachlässigbar.
Das Hochfrequenzfeld lässt sich somit durch
\begin{equation*}
  B_\text{x} = B_1 \cos\!\left(\omega t\right) \quad\quad
  B_\text{y} = B_1 \sin\!\left(\omega t\right)
\end{equation*}
darstellen.
Um die Blochschen Differentialgleichungen \ref{eq:Bloch-Gleichungen} zu Lösen, wird das Sytem in ein
Koordinatensystem transformiert, das mit der Frequenz $\omega$ um $\vec{B}_0$
rotiert. Im neuen System $\left\{x', y', z'\right\}$
ist $\vec{B}_\text{HF}$ zwar konstant (hier in $x'$-Richtung),
jedoch sind die Einheitsvektoren zeitabhängig.
Durch das Einführen eines effektiven Magnetfeldes
\begin{equation}
  \vec{B}_\text{eff} = \vec{B}_0 + \vec{B}_1 + \frac{\vec{\omega}}{\gamma}
\end{equation}
lässt sich die zeitliche Änderung der Probenmagnetisierung durch
\begin{equation}
  \frac{\symup{d} \vec{M}}{\symup{d} t} =
  \gamma \left(\vec{M} \times \vec{B}_\text{eff}\right)
\end{equation}
darstellen, was eine Präzession von $\vec{M}$ um die Feldrichtung $\vec{B_{\text{eff}}}$ beschreibt.
Im Resonanzfall ist die ist die Frequenz des Hochfrequenzfeldes die Larmorfrequenz und die Magnetisierung
präzessidiert mit einem Öffnungswinkel von 90° um die $\vec{B_1}$-Achse.
Wird das Hochfrequenzfeld für die Zeit
\begin{equation}
  \Delta t_{90} = \frac{\pi}{2 \gamma B_1}
  \quad\quad (\text{mit } \Delta t_{90} \ll T_1, T_2)
  \label{eq:t90}
\end{equation}
eingeschaltet, dreht sich die Magnetisierung aus der $z$-Richtung in die
$y$-Richtung.
Ebenso lässt sich ein 180°-Puls realisieren, der die Magnetisierung
in die negative $z$-Richtung dreht.
\subsection{Methoden zur Messung der Relaxationszeiten}
\paragraph{Der freie Induktionsfall}
Wird die Probenmagnetisierung um 90° aus der Gleichgewichtslage entfernt, präzediert
sie in der Ebene senkrecht dazu. Durch die Relaxationsprozesse kehrt die Magnetisierung
in die Ausgangslage. Dies wird freier Induktionsfall genannt und liegt an
dem statischen Magnetfeld $\vec{B_0}$. Dies hat nicht für alle Spins in einer realen Probe
den selben Betrag. Dies kann einerseits durch den Einfluss der Dipolfelder der benachbarten Spins kommen,
oder daran liegen, das ein reales Feld zwangslaufig inhomogen ist. Daraus resultiert eine
Verteilung von Larmorfrequenzen für die einzelnen Spins, was zu einer Dephasierung der
Spins untereinander und infolge dessen
zu einem Zerfall der transversalen Magnetisierung führt. Aus diesem Grund kommt zu der
transversale Zeitkonstante $T_2$ noch eine weitere Zeitkonstante $T_{\Delta B}$ hinzu, die
aus der apperativen inhomogenität des Magnetfelds entsteht. Dies macht den
freien Induktionsfall häufig kompliziert zu beschreiben, und die Brechnung von $T_2$ ist nur
für $T_{\Delta B} \ll T_2$ möglich.
