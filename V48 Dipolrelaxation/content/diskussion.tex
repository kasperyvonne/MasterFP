\section{Diskussion}
\label{sec:Diskussion}
Alle relativen Abweichungen wurden nach der Formel
\begin{equation*}
  \tilde{x} = \frac{ \lvert x_{lit} - x_{mess} \rvert}{\lvert x_{lit} \rvert}
  \cdot 100 \%
\end{equation*}
berechnet. 
In \cite{paper} wurde die charakteristische Relaxationszeit und Aktivierungsenergie von 
Strontiumionen in Kaliumbromid gemessen, dabei ergab sich eine charakteristische Relaxationszeit 
von \SI{4(2)e-14}{\second} und eine Aktivierungsenergie von 
$\SI{0.66(1)}{\eV} = \SI{1.06(2)e-19}{\joule}$.
Die Integralmethode ermöglicht die charakteristische Relaxationszeit zu bestimmen, diese wurde 
für die Heizrate \SI{2}{\celsius\per\minute} zu \SI{3.8e+19}{\second} und für die Heizrate 
\SI{1.5}{\celsius\per\minute} zu \SI{4.5e+17}{\second} bestimmt. Die Werte liegen mehrere 
Größenordnung auseinander, weshalb sich eine Bestimmung der relativen Abweichungen erübrigt.\\
Die Aktivierungsenergie wurde für die Heizrate \SI{2}{\celsius\per\minute} mit der Nährungsmethode 
zu \\\SI{1.5(1)e-19}{\joule} und mit der Integralmethode zu \SI{1.81(7)e-19}{\joule} bestimmt. 
Daraus ergibt sich eine relative Abweichung vom Literaturwert von \SI{45}{\percent} 
für die Nährungsmethode und \SI{71}{\percent} für die Integralmethode. Für die Heizrate 
\SI{1.5}{\celsius\per\minute} ergab sich \SI{1.5(1)e-19}{\joule} und \SI{1.66(5)e-19}{\joule}. 
Damit ergibt sich eine relative Abweichung vom Literaturwert von \SI{48}{\percent} und 
\SI{57}{\percent} für die Integramethode. Es zeigt sich, dass die Werte der Integralmethode 
sehr stark von den Literaturwerten abweichen. Die Integrationsmethode ist sehr 
empfindlich im Bezug auf die Abweichung der Werte aufgrund der vielen numerischen Mittel, 
die gebraucht werden, um diese zu verwirklichen.\\
Abweichungen werden dadurch erklärt, dass die Heizrate nicht immer konstant gehalten werden kann. 
Besonders die ersten aufgenommenen Werte sind davon betroffen, da sich die Probe besonders schnell 
erwärmt, wenn der Kühlfinger aus dem Stickstoff genommen wird. Zudem wird ein sehr geringer Strom 
gemessen, der durch magnetische und elektrische Felder gestört werden kann. 
Ist der Kondensator der in 
der Apparatur verbaut ist nicht vollständig entladen, kommt es deshalb zu systematischen Fehlern. 

