\section{Diskussion}
\label{sec:Diskussion}
\subsection{Relativer Fehler}
Alle relativen Fehler wurden nach der Formel
\begin{equation*}
  \tilde{x} = \frac{ \lvert x_{lit} - x_{mess} \rvert}{\lvert x_{lit} \rvert}
  \cdot 100 \%
\end{equation*}
berechnet, dabei bezeichnet $x_{lit}$ den Literaturwert der Messgröße $x_{mess}$.
\subsection{Relaxationszeiten \texorpdfstring{$T_1 \; \& \; T_2$}{math}}
Es fällt auf, dass die Werte von T1 und T2 nicht gleich groß sind. Es gilt $T_1 > T_2$ für beide 
ist der Literaturwert von \SI{2,5}{\second} gegeben \cite{hyper}. 
Die Quelle impliziert, dass dies üblich ist. Als 
relative Abweichung ergibt sich für $T_1$ \SI{8,5}{\percent} und für $T_2$ \SI{34,2}{\percent}. Da es üblicherweise auftritt, 
dass $T_2$ kleiner ist, sollte um eine Aussage über die Qualität der Messung zu machen, der relative Fehler von $T_1$ 
betrachtet werden. Dieser ist hinreichend genau unter den gegeben Umständen. Fehler sind damit zu erklären, dass die 
Befüllmenge der Probe nicht der, der vorgesehenen Halterung entspricht. Zudem können durch bewegen der Probe Störungen 
entstehen. Andere systematische Fehler könnten durch eine Verunreinigung der Probe oder Fluktuationen bzw. 
Inhomogenitäten der Temperatur und des B-Feldes auftreten und erklärt werden. Außerdem ist zu berücksichtigen, 
dass die Relaxationszeiten B-Feld abhängig sind und die Relativer der Quelle mit \SI{1}{\tesla} bestimmt wurden. 
Das hier verwendete B-Feld ist geringer. 
\subsection{Viskosität und Diffusionskonstante}
Die Temperatur wurde bei der Messung und der Literaturwerten mit \SI{20}{\celsius} angenommen. 
Der Literaturwert für die Viskosität für Wasser ist gegeben durch \SI{0.001}{\kilo\gram\per\meter\per\second} 
\cite{spekvis} und für die Diffusionskonstante durch \SI{2.023e-09}{\meter\per\second} \cite{Diff} gegeben. 
Daraus ergibt sich eine relative Abweichung von ca. \SI{6}{\percent} für die Viskosität und 
\SI{41}{\percent} für die Diffusionskonstante. Die Messung der Viskosität ist hinreichend genau, da hier 
größtenteils Fehler aufgrund der menschlichen Reaktionszeit entstehen, da nur mit einer Stoppuhr gemessen wird. 
Außerdem wurde die Messung nur einmal durchgeführt. Die Messung der Diffusionskonstante dagegen scheint Fehler 
behaftet zu sein. Die Methoden mit der diese Messung durchgeführt wurde sind ähnlich der mit den die 
Relaxationszeiten bestimmt wurden, daher sind hier auch die gleichen Fehlerquellen aufzuführen.
\subsection{Vergleich der Molekülradien}
Zum Vergleich der Molekülradien werden die Werte der Modelle mit dem der Stokschen Formel verglichen. 
Der Vergleich mit $r_{hcp}$ liefert eine relative Abweichung von \SI{10}{\percent} und 
mit $r_{krit}$ eine Abweichung von 
\SI{33}{\percent}. Abweichungen sind insbesonderen durch die hohe Abweichung der Diffusionskonstante 
zu erklären. Zudem sollte darauf verwiesen werden, dass es sich hier bei nur, um Modelle handelt mit denen 
der Molekülradius nur schätzungsweise berechnet wurde. 
