\section{Durchführung}
\label{sec:Durchführung}
Zuerst wird das Interferometer wie in Abbildung \ref{fig:Aufbau} gezeigt aufgebaut und die Spiegel justiert.
Ist die Apparatur justiert kann sie mit der Messvorrichtung wie in Kapitel \ref{sec:SI} geschildert erweitert
und vollendet werden. An den Dioden ist eine Messvorrichtung angebracht die bei einem Wechsel der Intensitäten
auf beiden Dioden einen Zähler aktiviert. So können die Interferenzmaxima bzw -minima gezählt werden.\\
Zur Messung des Kontrastes wird vor dem ersten PBSC ein Polarisationsfilter eingebracht und in den Strahlengang nachdem PBSC
das dünne Plättchen. Eine der Dioden wird jetzt an ein Oszilloskop angeschlossen.
Es wird ein Laser der Wellenlänge $\lambda_{vac}= 623,99\,$nm verwendet \cite{Anleitung}. Nun wird der
Polarisationswinkel in 5° Schritten variiert.Für feste Polarisationswinkel wird dann der Winkel des Plättchens im Strahlengang variiert.
Folglich kommt es jetzt zur Interferenz. Das Signal der Diode zeigt also den Intensitätswechel in Abhängigkeit vom
Winkel des Plättchens. Dieser wird nun mittels des Oszilloskops vermessen, daraus wird die maximale und minimale
Intensität gewonnen, woraus der Kontrast berechnet werden kann. Dann wird der Polarisationswinkel für
den maximalen Kontrast eingestellt und das Signal der Diode wird wieder auf den Zähler gegeben. \\
Nun wird wieder der Winkel des Plättchens variiert, dieses mal wird aber die Anzahl der
Interferenzmaxima in Abhängigkeit des Winkels gemessen. Das wurde sechsmal durchgeführt. \\
Nun wird das Plättchen aus dem Strahlengang entfernt und die Gaszelle eingebracht.
Die Gaszelle wird mit einer Vakuumpumpe evakuiert und dann langsam wieder mit Luft befüllt, dies
geschieht alles über ein Ventilsystem. Dabei wird die Anzahl der Interferenzmaxima in Abhängigkeit vom
Druck gemessen. Die Druckmessung geschieht über ein Piezoelement innerhalb der Gaszelle. Die Messung wurde
viermal durchgeführt.
